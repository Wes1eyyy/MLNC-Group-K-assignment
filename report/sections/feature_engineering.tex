% Feature Engineering Section

%% Development Plan for Feature Engineering:

%% 1. Overview (1 paragraph)
%%    - Importance of feature engineering in sports prediction
%%    - Goal: capture team performance, form, and match context
%%    - Overview of feature categories created

%% 2. Team Performance Features (subsection)
%%    2.1 Historical Statistics
%%        - Overall win/loss/draw rates
%%        - Home and away performance separately
%%        - Average goals scored and conceded
%%        - Shot statistics (total, on target, accuracy)
%%        - Discipline metrics (fouls, yellow/red cards)
%%        - Corner kicks
%%    
%%    2.2 Rationale
%%        - Why these features matter
%%        - Expected predictive power

%% 3. Form-Based Features (subsection)
%%    3.1 Rolling Window Statistics
%%        - Recent form (last 5, 10 matches)
%%        - Rolling averages: goals, points, wins
%%        - Trend indicators (improving vs declining)
%%    
%%    3.2 Implementation Details
%%        - Window sizes chosen
%%        - How to handle early matches (insufficient history)
%%        - Avoiding data leakage
%%    
%%    3.3 Rationale
%%        - Recent form often more predictive than overall statistics
%%        - Captures momentum and current team state

%% 4. Head-to-Head Features (subsection)
%%    4.1 Matchup-Specific Statistics
%%        - Historical performance between two specific teams
%%        - Win rates in H2H matches
%%        - Average goals in matchups
%%        - Last meeting outcome
%%    
%%    4.2 Rationale
%%        - Some teams have favorable/unfavorable matchups
%%        - Psychological and tactical factors

%% 5. Time-Based Features (subsection)
%%    5.1 Temporal Variables
%%        - Season stage (early/mid/late)
%%        - Month of year
%%        - Day of week (if relevant)
%%        - Days since last match (rest/fatigue)
%%    
%%    5.2 Rationale
%%        - Seasonality effects
%%        - Fixture congestion impact

%% 6. Advanced Features (subsection, optional)
%%    - Goal difference (cumulative)
%%    - Attack strength vs defense strength
%%    - Shot efficiency metrics
%%    - ELO ratings (if implemented)

%% 7. Feature Selection (subsection)
%%    7.1 Correlation Analysis
%%        - Feature correlation with target
%%        - Multicollinearity detection
%%        - Redundant feature removal
%%    
%%    7.2 Importance-Based Selection
%%        - Initial feature importance from tree models
%%        - Selected feature subset
%%    
%%    7.3 Final Feature Set
%%        - List of features used in final model
%%        - Justification for inclusion

%% 8. Feature Scaling (subsection)
%%    - Standardization or normalization approach
%%    - When and why applied
%%    - Impact on model performance

%% Expected Length: 2-3 pages

%% Key Points:
%% - Justify each feature category
%% - Explain how features avoid data leakage
%% - Include feature importance visualization
%% - Be specific about implementation
%% - Connect features to domain knowledge

%% Example Structure:
% \section{Feature Engineering}
%
% \subsection{Overview}
% TODO: Introduce feature engineering approach
%
% \subsection{Team Performance Features}
% TODO: Describe historical statistics features
%
% \subsection{Form-Based Features}
% TODO: Explain rolling window features
%
% \subsection{Head-to-Head Features}
% TODO: Describe matchup-specific features
%
% \subsection{Feature Selection}
% TODO: Explain selection process and final feature set
%
% % Include table of final features
% \begin{table}[h]
% \centering
% \caption{Final Feature Set}
% TODO: Add table of features with descriptions
% \end{table}
